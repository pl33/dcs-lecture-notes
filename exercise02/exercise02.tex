\phantomsection
\addcontentsline{toc}{section}{Exercise 2}
\section*{Exercise 2}

\begin{question}[subtitle={Mono-chromatic Signals}]
	Given is a mono-chromatic signal $u(t)$:
	\begin{equation*}
		u(t) = \SI{2}{V} \cdot \cos\left(2 \pi \cdot \SI{1}{MHz} \cdot t + \frac{\pi}{2} \right)
	\end{equation*}
	\begin{tasks}
		\task
		How much is the frequency and angular frequency? How much is the amplitude? How much is the phase?
		\task
		Give the phasor of the signal!
		\task
		An DC bias is added to the signal $u(t)$.
		\begin{equation*}
			u_2(t) = \SI{1}{V} + \SI{2}{V} \cdot \cos\left(2 \pi \cdot \SI{1}{MHz} \cdot t + \frac{\pi}{2} \right)
		\end{equation*}
		Is the resulting signal $u_2(t)$ still mono-chromatic?
	\end{tasks}
\end{question}

\begin{solution}
	\begin{tasks}
		\task
		\begin{itemize}
			\item Frequency: \SI{1}{MHz}
			\item Angular frequency: $2 \pi \cdot \SI{1}{MHz} = \SI{6283185.3}{s^{-1}}$
			\item Phase: $\SI{-\pi/2}{rad}$ or \SI{-90}{\degree}
			\item Amplitude: \SI{2}{V}
		\end{itemize}
		\task
		$\underline{U} = \SI{2}{V} \cdot e^{+j \frac{\pi}{2}}$ or $\underline{U} = \SI{2}{V} \angle +\frac{\pi}{2}$
		\task
		No, the DC bias adds a mono-chromatic component with a frequency of $f = 0$. $u_2(t)$ is a Fourier series.
	\end{tasks}
\end{solution}

\begin{question}[subtitle={Using the Fourier transform}]
	Derive the Fourier transform, without using the duality, of
	\begin{tasks}
		\task
		Derive the Fourier transform of the time shift, without using the duality!
		\begin{equation*}
			\mathcal{F}\left\{\underline{f}(t - t_0)\right\}
		\end{equation*}
		
		\task
		Derive the Fourier transform of the frequency shift, without using the duality!
		\begin{equation*}
			\mathcal{F}\left\{e^{j \omega_0 t} \underline{f}(t)\right\}
		\end{equation*}
		
		%\task
		%Derive the Fourier transform of the frequency shift using the time shift and duality!
	\end{tasks}
\end{question}

\begin{solution}
	\begin{tasks}
		\task
		Let
		\begin{equation*}
			\underline{h}(t) = \underline{f}(t - t_0)
		\end{equation*}
		The Fourier transform:
		\begin{equation*}
			\mathcal{F}\left\{\underline{h}(t)\right\} = \int\limits_{t = -\infty}^{\infty} \underline{f}(t - t_0) \cdot e^{-j \omega t} \, \mathrm{d} t
		\end{equation*}
		Substitute $t' = (t - t_0)$ in the integral.
		\begin{equation*}
			\mathcal{F}\left\{\underline{h}(t)\right\} = \int\limits_{t' = -\infty}^{\infty} \underline{f}(t') \cdot e^{-j \omega (t' + t_0)} \, \mathrm{d} t'
		\end{equation*}
		$e^{-j \omega t_0}$ is a constant.
		\begin{equation*}
			\mathcal{F}\left\{\underline{h}(t)\right\} = e^{-j \omega t_0} \underbrace{\int\limits_{t' = -\infty}^{\infty} \underline{f}(t') \cdot e^{-j \omega t'} \, \mathrm{d} t'}_{= \mathcal{F}\left\{\underline{f}(t)\right\} }
		\end{equation*}
		
		\task
		Let
		\begin{equation*}
			\underline{h}(t) = e^{j \omega_0 t} \underline{f}(t)
		\end{equation*}
		The Fourier transform:
		\begin{equation*}
			\mathcal{F}\left\{\underline{h}(t)\right\} = \int\limits_{t = -\infty}^{\infty} e^{j \omega_0 t} \underline{f}(t) \cdot e^{-j \omega t} \, \mathrm{d} t
		\end{equation*}
		Factor out $j t$ in the $e$-function.
		\begin{equation*}
			\mathcal{F}\left\{\underline{h}(t)\right\} = \int\limits_{t = -\infty}^{\infty} \underline{f}(t) \cdot e^{-j (\omega - \omega_0) t} \, \mathrm{d} t
		\end{equation*}
		Substitute $\omega' = \omega - \omega_0$ in the integral.
		\begin{equation*}
			\mathcal{F}\left\{\underline{h}(t)\right\} = \underbrace{\int\limits_{t = -\infty}^{\infty} \underline{f}(t) \cdot e^{-j \omega' t} \, \mathrm{d} t}_{= \mathcal{F}\left\{\underline{f}(t)\right\}}
		\end{equation*}
		\begin{equation*}
			\mathcal{F}\left\{\underline{h}(t)\right\} = \underline{F}\left(j \omega' \right) = \underline{F}\left(j \left(\omega - \omega_0\right) \right)
		\end{equation*}
		
%		\task
%		Let
%		\begin{equation*}
%			\underline{g}(t) = \underline{f}(t - t_0)
%		\end{equation*}
%		We know from a) that
%		\begin{equation*}
%			\underline{G}\left(\omega \right) = \mathcal{F}\left\{\underline{g}(t)\right\} = e^{-j \omega t_0} \cdot \underline{F}\left(\omega \right)
%		\end{equation*}
%		Now, swap $\omega$ and $t$, swap $t_0$ and $\frac{\omega_0}{2 \pi}$, and assume both $\underline{G}$ and $\underline{F}$ are time-domain functions from now on. $\underline{F}$ now represents the original time-domain function which is shifted in frequency.
%		\begin{equation*}
%			\underline{G}\left(t\right) = e^{- j \frac{\omega_0}{2 \pi} t} \cdot \underline{F}\left(t \right)
%		\end{equation*}
%		We already know $\underline{g}$. Assume that both $\underline{g}$ and $\underline{f}$ are frequency-domain functions now. Therefore, swap $\omega$ and $t$, ans swap $t_0$ and $\frac{2 \pi}{\omega_0}$, too.
%		\begin{equation*}
%			\mathcal{F}\left\{\underline{G}(t)\right\} = 2 \pi \cdot \underline{g}\left(- \omega\right) = \underline{f}\left(- \omega + \omega_0\right) = \underline{f}\left(\omega - \omega_0\right)
%		\end{equation*}
%		
%		We obtain the same result as in b). The duality works. \acs{QED}
	\end{tasks}
\end{solution}
