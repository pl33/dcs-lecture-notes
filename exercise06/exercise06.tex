% SPDX-License-Identifier: CC-BY-SA-4.0
%
% Copyright (c) 2020 Philipp Le
%
% Except where otherwise noted, this work is licensed under a
% Creative Commons Attribution-ShareAlike 4.0 License.
%
% Please find the full copy of the licence at:
% https://creativecommons.org/licenses/by-sa/4.0/legalcode

\phantomsection
\addcontentsline{toc}{section}{Exercise 6}
\section*{Exercise 6}


%%%%%%%%%%%%%%%%%%%%%%%%%%%%%%%%%%%%%%%%%%%%%%%%%%%%%%%%%%%%%%%%%%%%%%%%%%%%%%%
\begin{question}[subtitle={IIR Filter}]
	The following IIR filter is given.
	\begin{figure}[H]
		\centering
		\begin{circuitikz}
			\draw[o-] (-1,0) node[left, align=right]{$\underline{x}[n]$} -- (0,0);
			\draw (0,-3) node[adder](Add1){};
			\draw (2,-3) node[adder](Add2){};
			\draw (0,0) to[amp,l=$\underline{b}_0$,>,-] (Add1.north) node[inputarrow,rotate=-90]{};
			\draw (0,0) to[short,*-] (2,0) to[twoport,t=$z^{-1}$,>,-] (Add2.north) node[inputarrow,rotate=-90]{};
			\draw (Add1.east) to[short] (Add2.west) node[inputarrow,rotate=0]{};
			\draw[-latex] (Add2.east) to[short] (4,-3) node[right, align=left]{$\underline{y}[n]$};
			\draw (3,-3) to[short,*-] (3,-6) to[twoport,t=$z^{-1}$,>,-] (0,-6) to[amp,l=$\underline{a}_0$,>,-] (Add1.south) node[inputarrow,rotate=90]{};
		\end{circuitikz}
	\end{figure}
	with:
	\begin{itemize}
		\item $\underline{a}_0 = 0.5$
		\item $\underline{b}_0 = 2$
	\end{itemize}
	
	\begin{tasks}
		\task
		Give the block diagram of the filter!
		\task
		Give the differential equation of the filter!
		\task
		How much is the filter order?
		\task
		Is the filter stable?
		\task
		Plot the amplitude and phase response between $0$ and $\pi$.
	\end{tasks}
\end{question}

\begin{solution}
	\begin{tasks}
	\end{tasks}
\end{solution}


%%%%%%%%%%%%%%%%%%%%%%%%%%%%%%%%%%%%%%%%%%%%%%%%%%%%%%%%%%%%%%%%%%%%%%%%%%%%%%%
\begin{question}[subtitle={FIR Filter}]
	An FIR filter with following coefficients is given.
	\begin{itemize}
		\item $b_0 = 1$.
		\item $b_1 = 0.5 + j \cdot 1$.
		\item $b_2 = 2$.
	\end{itemize}

	The sampling rate of the digital system is \SI{2}{MHz}.
	
	\begin{tasks}
		\task
		Give the block diagram of the filter!
		\task
		Give the transfer function of the filter!
		\task
		Give the differential equation of the filter!
		\task
		How much is the filter order?
		\task
		Plot the amplitude and phase response between \SI{-1}{MHz} and \SI{1}{MHz}.
		\task
		Proof mathematically that all poles of the FIR filter are $0$!
	\end{tasks}
\end{question}

\begin{solution}
	\begin{tasks}
	\end{tasks}
\end{solution}


%%%%%%%%%%%%%%%%%%%%%%%%%%%%%%%%%%%%%%%%%%%%%%%%%%%%%%%%%%%%%%%%%%%%%%%%%%%%%%%
\begin{question}[subtitle={Down-sampling}]
	An analogue signal $x(t)$ is digitized (sampled and quantized).
	\begin{equation*}
		x(t) = \sin\left(2 \pi \cdot \SI{96}{kHz} \cdot t\right)
	\end{equation*}
	The signal has been sampled by a \SI{8}{bit}-ADC at \SI{7.68}{MHz}.
	
	The signal $x[n]$ is decimated by $N = 40$.
	
	\begin{tasks}
		\task
		How much is the sampling rate of the decimated signal?
		\task
		Is the signal suitable to be decimated by $N = 40$? Explain why! What is the criterion?
		\task
		What is the optimal sampling phase?
		\task
		Explain the effect on the spectrum caused by down-sampling!
		\task
		How much is the processing gain? How much is the effective number of bits?
	\end{tasks}
\end{question}

\begin{solution}
	\begin{tasks}
	\end{tasks}
\end{solution}


%%%%%%%%%%%%%%%%%%%%%%%%%%%%%%%%%%%%%%%%%%%%%%%%%%%%%%%%%%%%%%%%%%%%%%%%%%%%%%%
\begin{question}[subtitle={FFT}]
	A series of the samples in the time-domain is given:
	\begin{equation*}
		x[n] = \left[2 \underline{-0.5} 1 -2 \right]
	\end{equation*}
	
	\begin{remark}
		The underline marks the sample at $n = 0$.
	\end{remark}
	
	\begin{tasks}
		\task
		Calculate the DFT for $k = 0, \ldots, 3$!
		\task
		Calculate the FFT using the Cooley-Tuckey FFT algorithm!
		\task
		Compare the number of multiply-accumulate operations necessary for both methods in a) and b)!
		\task
		Draw the butterfly graph!
		\task
		Give th primitive roots of unity for each sub-FFT in the butterfly graph!
	\end{tasks}
\end{question}

\begin{solution}
	%The signal is periodic with $N = 4$.
	\begin{tasks}
	\end{tasks}
\end{solution}

%%%%%%%%%%%%%%%%%%%%%%%%%%%%%%%%%%%%%%%%%%%%%%%%%%%%%%%%%%%%%%%%%%%%%%%%%%%%%%%
%\begin{question}[subtitle={Decibel}]
%	\begin{tasks}
%	\end{tasks}
%\end{question}
%
%\begin{solution}
%	\begin{tasks}
%	\end{tasks}
%\end{solution}
