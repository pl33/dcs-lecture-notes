\chapter{Sampling and Time-Discrete Signals and Systems}

\begin{refsection}

\section{Time-Discrete Signals}

\subsection{Ideal Sampling}

% TODO
\begin{equation}
	\begin{split}
		\underline{x}[n] &= \int\limits_{-\infty}^{\infty} \underline{x}(t) \cdot \delta\left(t - n T_S\right) \, \mathrm{d} t \\
		 &= \underline{x}\left(n T_S\right)
	\end{split}
\end{equation}

\subsection{Discrete-Time Fourier Transform}

% TODO
\begin{equation}
	\underline{x}_S(t) = \sum\limits_{n = -\infty}^{\infty} \underline{x}[n] \cdot \delta(t - n T_S)
\end{equation}

\begin{equation}
	\begin{split}
		\underline{X}_S \left(j \omega\right) &= \mathcal{F} \left\{\underline{x}_S(t)\right\} \\
		 &= \mathcal{F} \left\{\sum\limits_{n = -\infty}^{\infty} \underline{x}[n] \cdot \delta(t - n T_S)\right\} \\
		 &= \int\limits_{t = -\infty}^{\infty} \sum\limits_{n = -\infty}^{\infty} \underline{x}[n] \cdot \delta(t - n T_S) \cdot e^{-j \omega t} \, \mathrm{d} t \\
		 &= \sum\limits_{n = -\infty}^{\infty} \int\limits_{t = -\infty}^{\infty} \underline{x}[n] \cdot \delta(t - n T_S) \cdot e^{-j \omega t} \, \mathrm{d} t \\
		 &= \sum\limits_{n = -\infty}^{\infty} \underline{x}[n] \cdot e^{-j \omega n T_S}
	\end{split}
\end{equation}

Redefining $\phi = T_S \omega$:
\begin{equation}
	\underline{X}_S \left(j \omega\right) = \underline{X} \left(e^{j \phi}\right) = \sum\limits_{n = -\infty}^{\infty} \underline{x}[n] \cdot e^{-j \phi n}
\end{equation}

\subsection{Sampling Theorem and Aliasing}

\subsection{Discrete Fourier Transform}

\section{Analogies Of Time-Continuous and Time-Discrete Signals and Systems}

\subsection{Transforms}

\begin{table}[H]
	\centering
	\begin{tabular}{|p{0.3\linewidth}||p{0.3\linewidth}|p{0.3\linewidth}|}
		\hline
		{} & \textbf{Frequency-Continuous Domain} & \textbf{Frequency-Discrete Domain} \\
		\hline
		\hline
		\textbf{Time-Continuous Domain} & Fourier transform & Fourier series \\
		\hline
		\textbf{Time-Discrete Domain} & Discrete-Time Fourier transform & Discrete Fourier transform \\
		\hline
	\end{tabular}
\end{table}

\subsubsection{Obtaining a frequency-continuous domain:}

\begin{minipage}{0.45\linewidth}
	\textbf{From the time-continuous domain (analog signal):}
	
	\vspace{0.5em}
	
	Fourier transform:
	\begin{equation*}
		\underline{X}(j \omega) = \int\limits_{t = -\infty}^{\infty} \underline{x}(t) \cdot e^{-j \omega t} \, \mathrm{d} t
	\end{equation*}
	
	Inverse Fourier transform:
	\begin{equation*}
		\underline{x}(t) = \frac{1}{2 \pi} \int\limits_{\omega = -\infty}^{\infty} \underline{X}(j \omega) \cdot e^{+ j \omega t} \, \mathrm{d} \omega
	\end{equation*}
	
	\begin{itemize}
		\item Continuous time: $t \in \mathbb{R}$
		\item Continuous frequency: $\omega \in \mathbb{R}$
	\end{itemize}
\end{minipage}
\hfill
\begin{minipage}{0.45\linewidth}
	\textbf{From the time-discrete domain (digital signal):}
	
	\vspace{0.5em}
	
	Discrete-time Fourier transform:
	\begin{equation*}
		\underline{X}_{2\pi}(e^{j \phi}) = \sum\limits_{n = -\infty}^{\infty} \underline{x}[n] \cdot e^{- j \phi n}
	\end{equation*}
	
	Inverse discrete-time Fourier transform:
	\begin{equation*}
		\underline{x}[n] = \frac{1}{2 \pi} \int\limits_{- \pi}^{+ \pi} \underline{X}_{2\pi}(e^{j \phi}) \cdot e^{+ j \phi n} \, \mathrm{d} \phi
	\end{equation*}
	
	\begin{itemize}
		\item Discrete time: $n \in \mathbb{Z}$
		\item Continuous frequency: $\phi \in \mathbb{R}$
	\end{itemize}
\end{minipage}

\subsubsection{Obtaining a frequency-discrete domain:}

\begin{minipage}{0.45\linewidth}
	\textbf{From the time-continuous domain (analog signal):}
	
	\vspace{0.5em}
	
	Fourier analysis:
	\begin{equation*}
		\underline{X}[k] = \frac{\omega_0}{2 \pi} \int\limits_{-\frac{T_0}{2}}^{\frac{T_0}{2}} \underline{x}(t) \cdot e^{-j k \omega_0 t} \, \mathrm{d} t
	\end{equation*}
	
	Fourier series:
	\begin{equation*}
		\underline{x}(t) = \sum\limits_{k = -\infty}^{\infty} \underline{X}[k] \cdot e^{+ j k \omega_0 t}
	\end{equation*}
	
	\begin{itemize}
		\item Continuous time: $t \in \mathbb{R}$
		\item Discrete frequency: $k \in \mathbb{Z}$
	\end{itemize}
\end{minipage}
\hfill
\begin{minipage}{0.45\linewidth}
	\textbf{From the time-discrete domain (digital signal):}
	
	\vspace{0.5em}
	
	Discrete Fourier transform:
	\begin{equation*}
		\underline{X}[k] = \sum\limits_{n = 0}^{N - 1} \underline{x}[n] \cdot e^{- j \frac{2 \pi}{N} k n}
	\end{equation*}
	
	Inverse discrete Fourier transform:
	\begin{equation*}
		\underline{x}[n] = \frac{1}{N} \sum\limits_{k = 0}^{N - 1} \underline{X}[k]  \cdot e^{+ j \frac{2 \pi}{N} k n}
	\end{equation*}
	
	\begin{itemize}
		\item Discrete time: $n \in \mathbb{Z}$
		\item Discrete frequency: $k \in \mathbb{Z}$
	\end{itemize}
\end{minipage}

\subsection{Systems}

\subsection{Cross-Correlation and Autocorrelation}

\subsection{Spectral Density}

\subsection{Noise}

\section{Digital Signals and Systems}

\subsection{Quantization}

\subsection{Quantization Error}

\subsection{Window Filters}

\subsection{Time Recovery}

\subsection{Practical Issues}

\phantomsection
\addcontentsline{toc}{section}{References}
\printbibliography[heading=subbibliography]
\end{refsection}

