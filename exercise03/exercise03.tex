% SPDX-License-Identifier: CC-BY-SA-4.0
%
% Copyright (c) 2020 Philipp Le
%
% Except where otherwise noted, this work is licensed under a
% Creative Commons Attribution-ShareAlike 4.0 License.
%
% Please find the full copy of the licence at:
% https://creativecommons.org/licenses/by-sa/4.0/legalcode

\phantomsection
\addcontentsline{toc}{section}{Exercise 3}
\section*{Exercise 3}

%%%%%%%%%%%%%%%%%%%%%%%%%%%%%%%%%%%%%%%%%%%%%%%%%%%%%%%%%%%%%%%%%%%%%%%%%%%%%%%
\begin{question}[subtitle={Stochastic Process}]
	A normally distributed random process produces the sequences $x_1(t)$, $x_2(t)$ and $x_3(t)$.
	\begin{table}[H]
		\centering
		\begin{tabular}{|l|r|r|r|r|r|r|r|r|r|r|}
			\hline
			$t$ & 0 & 1 & 2 & 3 & 4 & 5 & 6 & 7 & 8 & 9 \\
			\hline
			\hline
			$x_1(t)$ & 4.99 & 4.37 & 8.57 & 4.01 & 3.77 & 3.35 & 3.87 & 8.39 & 6.89 & 1.96 \\
			\hline
			$x_2(t)$ & 3.95 & 5.35 & 2.94 & 6.38 & 4.78 & 7.62 & 5.25 & 6.81 & 5.65 & 5.29 \\
			\hline
			$x_3(t)$ & 7.01 & 4.40 & 4.26 & 6.54 & 4.53 & 6.85 & 4.46 & 5.81 & 6.49 & 4.11 \\
			\hline
		\end{tabular}
	\end{table}
	\begin{tasks}
		\task
		Calculate the stochastic mean for each time instance!
		
		\task
		Calculate the temporal mean for each sequence!
		
		\task
		The process is ergodic with $\mu_x = 5.00$. However, why is the condition $\E\left\{\vect{x}\right\} = \overline{x} = \mu_x$ not fulfilled?
	\end{tasks}
\end{question}

\begin{solution}
	\begin{tasks}
		\task
		\begin{table}[H]
			\centering
			\begin{tabular}{|l|r|r|r|r|r|r|r|r|r|r|}
				\hline
				$t$ & 0 & 1 & 2 & 3 & 4 & 5 & 6 & 7 & 8 & 9 \\
				\hline
				\hline
				$\E\left\{\vect{x}(t)\right\}$ & 5.32 & 4.71 & 5.26 & 5.64 & 4.36 & 5.94 & 4.53 & 7.00 & 6.34 & 3.79 \\
				\hline
			\end{tabular}
		\end{table}
	
		\task
		\begin{itemize}
			\item $\overline{x_1} = 5.01$
			\item $\overline{x_1} = 5.40$
			\item $\overline{x_1} = 5.45$
		\end{itemize}
	
		\task
		\begin{itemize}
			\item There are only $N = 3$ sequences drawn from the random process. $\E\left\{\vect{x}\right\}$ will converge to $5.00$ for $N \rightarrow \infty$. $N = 3$ is too short.
			\item Each sequence is only $L = 10$ samples long. $\overline{x}$ will converge to $5.00$ for $L \rightarrow \infty$. $L = 10$ is too short.
		\end{itemize}
	\end{tasks}
\end{solution}

%%%%%%%%%%%%%%%%%%%%%%%%%%%%%%%%%%%%%%%%%%%%%%%%%%%%%%%%%%%%%%%%%%%%%%%%%%%%%%%
\begin{question}[subtitle={Cross-Correlation and Autocorrelation}]
	Two signals are given $f_1(t)$ and $f_2(t)$. The signals are value- and time-continuous. Ten samples are given. Both signals are zero for $t < -5$ and $t > 5$.
	\begin{table}[H]
		\centering
		\begin{tabular}{|l|r|r|r|r|r|r|r|r|r|r|r|}
			\hline
			$t$ & -5 & -4 & -3 & -2 & -1 & 0 & 1 & 2 & 3 & 4 & 5 \\
			\hline
			\hline
			$f_1(t)$ & 0 & 1 & 2 & 3 & 4 & 5 & 4 & 3 & 2 & 1 & 0 \\
			$f_2(t)$ & -1.34 & 0.30 & 14.54 & -1.54 & -3.03 & -1.72 & 14.16 & 2.70 & 1.17 & -2.44 & -4.66 \\
			\hline
		\end{tabular}
	\end{table}
	\begin{tasks}
		\task
		Calculate the cross-correlation
		\begin{equation*}
			\mathrm{R}_{f_1 f_2}(\tau) \approx \left(f_1 \star f_2\right)(\tau) = \sum\limits_{t=-5}^{5} f_1(t) f_2(t + \tau)
		\end{equation*}
		for $\tau = 0$, $\tau = 1$ and $\tau = 2$.
		
		\task
		Signal $f_2(t)$ contains signal $f_1(t)$ which is superimposed by noise. Using the values calculated in a), how much is the time $\Delta t$ lag between $f_1(t)$ and $f_2(t)$?
		
		\task
		Calculate the autocorrelation
		\begin{equation*}
			\mathrm{R}_{f_1 f_1}(\tau) \approx \left(f_1 \star f_1\right)(\tau) = \sum\limits_{t=-5}^{5} f_1(t) f_1(t + \tau)
		\end{equation*}
		for $\tau = -2$, $\tau = -1$, $\tau = 0$, $\tau = 1$ and $\tau = 2$.
		
		\task
		How much is the signal energy $E$ of $f_1(t)$?
		\begin{equation*}
			E \approx \frac{1}{T} \sum\limits_{t=-5}^{5} \left|f_1(t)\right|^2
		\end{equation*}
		with $T = 11$?
	\end{tasks}

	\textit{Remark:} Only samples of the functions with a spacing of $1$ are given. Therefore, the indefinite integrals of both cross-correlation and autocorrelation can be approximated using the sums.
\end{question}

\begin{solution}
	\begin{tasks}
		\task
		Full cross-correlation between $-10 \geq t \geq 10$. The cross-correlation is zero everywhere outside of the interval.
		\begin{table}[H]
			\centering
			\begin{tabular}{|l|r|r|r|r|r|r|r|r|r|}
				\hline
				$\tau$ & -10 & -9 & -8 & -7 & -6 & -5 & -4 & -3 & -2 \\
				\hline
				$\mathrm{R}_{f_1 f_2}(\tau)$ & $0.0$ & $-1.34$ & $-2.38$ & $11.12$ & $23.08$ & $32.01$ & $41.9$ & $65.35$ & $62.42$ \\
				\hline
				\hline
				$\tau$ & -1 & 0 & 1 & 2 & 3 & 4 & 5 & 6 & 7 \\
				\hline
				$\mathrm{R}_{f_1 f_2}(\tau)$ & $63.74$ & $68.68$ & $71.06$ & $45.42$ & $28.92$ & $8.54$ & $-9.99$ & $-20.92$ & $-17.69$ \\
				\hline
				\hline
				$\tau$ & 8 & 9 & 10 & & & & & & \\
				\hline
				$\mathrm{R}_{f_1 f_2}(\tau)$ & $-11.76$ & $-4.66$ & $0.0$ & & & & & & \\
				\hline
			\end{tabular}
		\end{table}
		\begin{itemize}
			\item $\left(f_1 \star f_2\right)(0) = 68.68$
			\item $\left(f_1 \star f_2\right)(1) = 71.06$
			\item $\left(f_1 \star f_2\right)(2) = 45.42$
		\end{itemize}
	
		\task
		\begin{itemize}
			\item The maximum value is at $\tau = 1$.
			\item $f_2(t)$ is delayed (negatively advanced) by $\Delta t = -1$ in relation to $f_1(t)$.
		\end{itemize}
	
		\task
		Full autocorrelation between $-10 \geq t \geq 10$. The autocorrelation is zero everywhere outside of the interval.
		\begin{table}[H]
			\centering
			\begin{tabular}{|l|r|r|r|r|r|r|r|r|r|r|r|}
				\hline
				$\tau$ & -10 & -9 & -8 & -7 & -6 & -5 & -4 & -3 & -2 & -1 & 0 \\
				\hline
				$\mathrm{R}_{f_1 f_1}(\tau)$ & 0 &  0 &  1 &  4 & 10 & 20 & 3. & 52 & 68 & 80 & 85 \\
				\hline
				\hline
				$\tau$ & 1 & 2 & 3 & 4 & 5 & 6 & 7 & 8 & 9 & 10 & \\
				\hline
				$\mathrm{R}_{f_1 f_1}(\tau)$ & 80 & 68 & 52 & 35 & 20 & 10 &  4 &  1 &  0 &  0 & \\
				\hline
			\end{tabular}
		\end{table}
		Only three values must be calculated:
		\begin{itemize}
			\item $\mathrm{R}_{f_1 f_1}(\tau)(-2) = 68$
			\item $\mathrm{R}_{f_1 f_1}(\tau)(-1) = 80$
			\item $\mathrm{R}_{f_1 f_1}(\tau)(0) = 85$
		\end{itemize}
		The other two values can be deducted from the symmetry rules:
		\begin{itemize}
			\item $\mathrm{R}_{f_1 f_1}(\tau)(1) = \mathrm{R}_{f_1 f_1}(\tau)(-1) = 80$
			\item $\mathrm{R}_{f_1 f_1}(\tau)(2) = \mathrm{R}_{f_1 f_1}(\tau)(-2) = 68$
		\end{itemize}
	
		\task
		\begin{itemize}
			\item It is not necessary to solve the sum again.
			\item The sum is $\mathrm{R}_{f_1 f_1}(\tau)(0)$.
			\item The energy is
			\begin{equation*}
				E = \frac{1}{11} \mathrm{R}_{f_1 f_1}(\tau)(0) = 7.23
			\end{equation*}
		\end{itemize}
	\end{tasks}
\end{solution}


%%%%%%%%%%%%%%%%%%%%%%%%%%%%%%%%%%%%%%%%%%%%%%%%%%%%%%%%%%%%%%%%%%%%%%%%%%%%%%%
\begin{question}[subtitle={Power Spectral Density}]
	Are the following statements about the power spectral density (PSD) true or false? If false, correct the statement!
	\begin{tasks}
		\task
		The PSD is measure for the fraction of power which is located at a certain frequency.
		
		\task
		The unit of the PSD is the same as the signal.
		
		\task
		The PSD is even, if the signal is purely real-valued.
		
		\task
		The PSD is odd, if the signal is complex-valued.
		
		\task
		The PSD is always real-valued.
		
		\task
		The PSD can be used to determine the output signal of an LTI system in amplitude and phase.
		
	\end{tasks}
\end{question}

\begin{solution}
	\begin{tasks}
		\task
		True. The signal power can be obtained by integration of the PSD over all frequencies.
		\begin{equation*}
			P = \int\limits_{-\infty}^{\infty} \mathrm{S}_{xx}(\omega) \; \mathrm{d} \omega
		\end{equation*}
		
		\task
		False.
		\begin{equation*}
			\left(\text{Unit of PSD}\right) = \frac{\left(\text{Unit of signal}\right)^2}{\left(\text{Unit of frequency}\right)}
		\end{equation*}
		
		For example \si{W/Hz}. Never \si{W} only, because it is the density of the power across the frequency.
		
		\task
		True
		
		\task
		False. There are no symmetry rules for the PSD, if the signal is complex-valued.
		
		\task
		True.
		\begin{itemize}
			\item The autocorrelation function is always Hermitian.
			\item The Fourier transform of a Hermitian function is always real-valued.
			\item Therefore, the PSD is real-valued.
		\end{itemize}
		
		\task
		False.
		\begin{itemize}
			\item Only the amplitude can be calculated.
			\item The phase information of the system is lost by squaring the absolute value of the transfer funtion.
			\item Furthermore, the PSD does not contain phase information of the signal.
		\end{itemize}
	\begin{equation*}
		\mathrm{S}_{yy}(\omega) = \left|\underline{H}\left(j \omega\right)\right|^2 \cdot \mathrm{S}_{xx}(\omega)
	\end{equation*}
	\end{tasks}
\end{solution}


%%%%%%%%%%%%%%%%%%%%%%%%%%%%%%%%%%%%%%%%%%%%%%%%%%%%%%%%%%%%%%%%%%%%%%%%%%%%%%%
\begin{question}[subtitle={Decibel}]
	\begin{tasks}
		\task
		Convert to logarithmic scale or vice versa!
		\begin{enumerate}
			\item \SI{2}{mW}, $P_0 = \SI{1}{mW}$
			\item \SI{2}{V}, $U_0 = \SI{1}{V}$
			\item \num{400000}
			\item \SI{5}{GHz}, $f_0 = \SI{1}{Hz}$
			\item \SI{8}{fW/MHz}, $P_0 = \SI{1}{mW}$
			\item \SI{5}{dB\micro\volt}
			\item \SI{-10}{dBm}
		\end{enumerate}
	
		\task
		Convert to logarithmic scale or vice versa!
		\begin{enumerate}
			\item $1$
			\item $2$
			\item $4$
			\item $0.5$
			\item $0.25$
			\item $10$
			\item $100$
			\item $0.1$
			\item $0.01$
			\item $500$
			\item $0.04$
		\end{enumerate}
	
		\task
		A signal is transmitted with a power of \SI{13}{dBm}. The signal is attenuated by \SI{86}{dB} on its way to the receiver. Assume that a ohmic resistance of \SI{50}{\ohm} is connected to the antenna. What voltage can be measured at this resistance? Give your result in linear and logarithmic scale!
	\end{tasks}
\end{question}

\begin{solution}
	\begin{tasks}
		\task
		\begin{enumerate}
			\item $\SI{10}{dBm} \cdot \log_{10} \left(\frac{\SI{2}{mW}}{\SI{1}{mW}}\right) = \SI{3}{dBm}$
			\item $\SI{20}{dBV} \cdot \log_{10} \left(\frac{\SI{2}{V}}{\SI{1}{V}}\right) = \SI{6}{dBV}$ (20 instead of 10 for voltages and currents)
			\item $\SI{10}{dB} \cdot \log_{10} \left(400000\right) = \SI{53}{dB}$
			\item $\SI{10}{dBHz} \cdot \log_{10} \left(\frac{\SI{5}{GHz}}{\SI{1}{Hz}}\right) = \SI{97}{dbHz}$
			\item $\SI{10}{dBm/Hz} \cdot \log_{10} \left(\frac{\SI{8}{fW/MHz}}{\SI{1}{mW/Hz}}\right) = \SI{10}{dBm/Hz} \cdot \log_{10} \left(\frac{10^{-6} \cdot \SI{8}{fW/Hz}}{\SI{1}{mW/Hz}}\right) = \SI{-171}{dBm/Hz}$
			\item $\SI{1}{\micro\volt} \cdot 10^{\SI{5}{dB\micro\volt} / 20} = \SI{1.8}{\micro\volt}$
			\item $\SI{1}{mW} \cdot 10^{\SI{-10}{dBm} / 10} = \SI{100}{\micro.W}$
		\end{enumerate}
	
		\task
		\begin{enumerate}
			\item $\SI{0}{dB}$
			\item $\SI{3}{dB}$
			\item $\SI{6}{dB}$, Hint: $4 = 2 \cdot 2 \equiv \SI{3}{dB} + \SI{3}{dB} = \SI{6}{dB}$
			\item $\SI{-3}{dB}$
			\item $\SI{-6}{dB}$, Hint: $0.25 = 0.5 \cdot 0.5 \equiv \SI{-3}{dB} + \SI{-3}{dB} = \SI{-6}{dB}$
			\item $\SI{10}{dB}$
			\item $\SI{20}{dB}$
			\item $\SI{-10}{dB}$
			\item $\SI{-20}{dB}$
			\item $\SI{27}{dB}$, Hint: $50 = 10 \cdot 10 \cdot 10 \cdot 0.5 \equiv \SI{10}{dB} + \SI{10}{dB} + \SI{10}{dB} - \SI{3}{dB} = \SI{27}{dB}$
			\item $\SI{-14}{dB}$, Hint: $0.04 = 0.1 \cdot 0.1 \cdot 2 \cdot 2 \equiv \SI{-10}{dB} + \SI{-10}{dB} + \SI{3}{dB} + \SI{3}{dB} = \SI{-14}{dB}$
		\end{enumerate}
		Remember these rules as an RF engineer! You will need them often. ;)
		
		\task
		Power at the receiver antenna:
		\begin{equation*}
			\SI{13}{dBm} \underbrace{- \SI{86}{dB}}_{\text{Atennuation}} = \SI{-73}{dBm}
		\end{equation*}
		
		\begin{itemize}
			\item 1st way: Convert to linear scale, then back to logarithmic scale
			\begin{equation*}
				\begin{split}
					\SI{-73}{dBm} &\equiv \SI{50.1}{pW} \\
					U = \sqrt{P \cdot R} &= \sqrt{\SI{50.1}{pW} \cdot \SI{50}{\ohm}} = \SI{50}{\micro\volt} \\
					\SI{50}{\micro\volt} &\equiv \underbrace{\SI{20}{dB\micro\volt}}_{\text{Attention!}} \cdot \log_{10}\left(\frac{\SI{50}{\micro\volt}}{\SI{1}{\micro\volt}}\right) = \SI{34}{dB\micro\volt}
				\end{split}
			\end{equation*}
			\item 2nd way: Convert everything to logarithmic scale. The multiplication of power and resistance must happen with ``pure'' SI units.
			\begin{equation*}
			\begin{split}
				\SI{50}{\ohm} &\equiv \SI{10}{dB\ohm} \cdot \log_{10} \left(\frac{\SI{50}{\ohm}}{\SI{1}{\ohm}}\right) = \SI{17}{dB\ohm} \\
				\SI{-73}{dBm} \underbrace{- \SI{30}{dB}}_{\text{\si{mW} to \si{W}}} &= \SI{-103}{dBW} \quad \text{SI unit!} \\
				\SI{-103}{dBW} + \SI{17}{dB\ohm} &= \SI{-86}{dBV} \quad \text{SI unit!} \\
				\SI{-86}{dBV} \underbrace{+ \SI{120}{dB}}_{\text{\si{V} to \si{\micro\volt}}} &= \SI{34}{dB\micro\volt} \\
				\SI{34}{dB\micro\volt} &\equiv \SI{1}{\micro\volt} \cdot 10^{\frac{34}{20}} = \SI{50}{\micro\volt}
			\end{split}
			\end{equation*}
		\end{itemize}
	
		\textit{Remarks:} Conversion of different SI prefixes:
		\begin{itemize}
			\item From \si{mW} to \si{W}:
			\begin{equation*}
				\begin{split}
					\SI{1}{mW} &= \SI{0.001}{W} \\
					\SI{10}{dB} \cdot \log_{10}\left(\frac{\SI{0.001}{W}}{\SI{1}{mW}}\right) &= \SI{-30}{dB}
				\end{split}
			\end{equation*}
			\item From \si{V} to \si{\micro\volt}:
			\begin{equation*}
				\begin{split}
					\SI{1}{V} &= \SI{1000000}{\micro\volt} \\
					\SI{20}{dB} \cdot \log_{10}\left(\frac{\SI{1000000}{\micro\volt}}{\SI{1}{V}}\right) &= \SI{120}{dB}
				\end{split}
			\end{equation*}
		\end{itemize}
	\end{tasks}
\end{solution}

%%%%%%%%%%%%%%%%%%%%%%%%%%%%%%%%%%%%%%%%%%%%%%%%%%%%%%%%%%%%%%%%%%%%%%%%%%%%%%%
\begin{question}[subtitle={Noise}]
	The following system is given:
	\begin{figure}[H]
		\centering
		\begin{adjustbox}{scale=0.7}
			\begin{tikzpicture}
				\node[draw, block] (BPF){Band-pass filter\\ $L_{G,1} = \SI{-3}{dB}$\\ $L_{F,1} = \SI{3}{dB}$};
				\node[draw, block, right=of BPF] (Mix){Mixer\\ $L_{G,2} = \SI{3}{dB}$\\ $L_{F,2} = \SI{20}{dB}$};
				\node[draw, block, right=of Mix] (Demod){Demodulator\\ $L_{G,3} = \SI{6}{dB}$\\ $L_{F,3} = \SI{10}{dB}$};
				\node[draw, block, right=of Demod] (Amp){Amplifier\\ $L_{G,4} = \SI{50}{dB}$\\ $L_{F,4} = \SI{16}{dB}$};
				
				\draw[o->] ([xshift=-1cm]BPF.west) node[left,align=right]{Input $u(t)$} -- (BPF.west);
				\draw[->] (BPF.east) -- (Mix.west);
				\draw[->] (Mix.east) -- (Demod.west);
				\draw[->] (Demod.east) -- (Amp.west);
				\draw[->] (Amp.east) -- ([xshift=1cm]Amp.east) node[right,align=left]{Output};
			\end{tikzpicture}
		\end{adjustbox}
	\end{figure}
	\begin{itemize}
		\item The band-pass filter has an input impedance of $\SI{50}{\ohm}$.
		\item The band-pass filter has a bandwidth of $\SI{2}{MHz}$.
		\item All system components are at room temperature: \SI{25}{\degreeCelsius}.
	\end{itemize}
	The input signal is:
	\begin{equation*}
		u(t) = \SI{70.71}{\micro\volt} \cdot \cos\left(\omega_0 t\right)
	\end{equation*}
	
	Give all results in the logarithmic scale!
	\begin{tasks}
		\task
		What is the signal power, the thermal noise power and the SNR at the input?
		
		\task
		What is the overall noise figure and overall gain of the chain? What is the signal power, the thermal noise power and the SNR at the output of the chain?
		
		\task
		Now a low noise amplifier is placed in front of the mixer. The gain of the last amplifier is reduced, so that the overall gain does not change.
		\begin{figure}[H]
			\centering
			\begin{adjustbox}{scale=0.5}
				\begin{tikzpicture}
					\node[draw, block] (Amp){Low noise amplifier\\ $L_{G,0} = \SI{30}{dB}$\\ $L_{F,0} = \SI{6}{dB}$};
					\node[draw, block, right=of Amp] (BPF){Band-pass filter\\ $L_{G,1} = \SI{-3}{dB}$\\ $L_{F,1} = \SI{3}{dB}$};
					\node[draw, block, right=of BPF] (Mix){Mixer\\ $L_{G,2} = \SI{3}{dB}$\\ $L_{F,2} = \SI{20}{dB}$};
					\node[draw, block, right=of Mix] (Demod){Demodulator\\ $L_{G,3} = \SI{6}{dB}$\\ $L_{F,3} = \SI{10}{dB}$};
					\node[draw, block, right=of Demod] (Amp2){Amplifier\\ $L_{G,4} = \SI{20}{dB}$\\ $L_{F,4} = \SI{16}{dB}$};
					
					\draw[o->] ([xshift=-1cm]Amp.west) node[left,align=right]{Input $u(t)$} -- (Amp.west);
					\draw[->] (Amp.east) -- (BPF.west);
					\draw[->] (BPF.east) -- (Mix.west);
					\draw[->] (Mix.east) -- (Demod.west);
					\draw[->] (Demod.east) -- (Amp2.west);
					\draw[->] (Amp2.east) -- ([xshift=1cm]Amp2.east) node[right,align=left]{Output};
				\end{tikzpicture}
			\end{adjustbox}
		\end{figure}
		For this new constellation: What is the overall noise figure and overall gain of the chain? What is the signal power, the thermal noise power and the SNR at the output of the chain?
	\end{tasks}
\end{question}

\begin{solution}
	\begin{tasks}
		\task
		\begin{itemize}
			\item The RMS value of the signals is: $\frac{\SI{70.71}{\micro\volt}}{\sqrt{2}} = \SI{50}{\micro\volt}$
			\item At $\SI{50}{\ohm}$, this yields a power of: $P = \frac{U^2}{R} = \SI{50}{pW} \equiv L_{P,S} = \SI{-73}{dBm}$
			\item The noise floor at room temperature is: $k_B T \equiv \SI{-174}{dBm/Hz}$
			\item The filter's bandwidth is: $\SI{63}{dBHz}$
			\item The noise power is: $k_B T \Delta f \equiv L_{P,N} = \SI{-174}{dBm/Hz} + \SI{63}{dBHz} = \SI{-111}{dBm}$
			\item The SNR is: $L_{\mathrm{SNR}} = L_{P,S} - L_{P,N} = \SI{-73}{dBm} - (\SI{-111}{dBm}) = \SI{38}{dBm}$
		\end{itemize}
	
		\task
		Convert to linear scale:
		\begin{itemize}
			\item $G_1 = 0.5$
			\item $F_1 = 2$
			\item $G_2 = 2$
			\item $F_2 = 100$
			\item $G_3 = 4$
			\item $F_3 = 10$
			\item $G_4 = 100000$
			\item $F_4 = 40$
		\end{itemize}
	
		Using the Friis formula:
		\begin{equation*}
			\begin{split}
				F_{\text{total}} &= F_1 + \frac{F_2 - 1}{G_1} + \frac{F_3 - 1}{G_1 G_2} + \frac{F_4 - 1}{G_1 G_2 G_3} \\
				 &= 218 \\
				L_{F,\text{total}} &= \SI{23.4}{dB}
			\end{split}
		\end{equation*}
		Overall gain:
		\begin{equation*}
			L_{G,\text{total}} = \SI{-3}{dB} + \SI{3}{dB} + \SI{6}{dB} + \SI{50}{dB} = \SI{56}{dB}
		\end{equation*}
		\begin{itemize}
			\item Output signal power: $\SI{-73}{dBm} + L_{G,\text{total}} = \SI{-17}{dBm}$
			\item Output noise power: $\SI{-111}{dBm} + L_{G,\text{total}} + L_{F,\text{total}} = \SI{-31.6}{dBm}$
			\item Output SNR: $\SI{38}{dBm} - L_{F,\text{total}} = \SI{14.6}{dB}$
		\end{itemize}
	
		\task
		Convert to linear scale:
		\begin{itemize}
			\item $G_0 = 1000$
			\item $F_0 = 4$
			\item $G_1 = 0.5$
			\item $F_1 = 2$
			\item $G_2 = 2$
			\item $F_2 = 100$
			\item $G_3 = 4$
			\item $F_3 = 10$
			\item $G_4 = 10$
			\item $F_4 = 40$
		\end{itemize}
	
		Using the Friis formula:
		\begin{equation*}
			\begin{split}
				F_{\text{total}} &= F_0 + \frac{F_1 - 1}{G_0} + \frac{F_2 - 1}{G_0 G_1} + \frac{F_3 - 1}{G_0 G_1 G_2} + \frac{F_4 - 1}{G_0 G_1 G_2 G_3} \\
				 &= 4.22 \\
				L_{F,\text{total}} &= \SI{6.3}{dB}
			\end{split}
		\end{equation*}
		Overall gain:
		\begin{equation*}
			L_{G,\text{total}} = \SI{50}{dB} + \SI{-3}{dB} + \SI{3}{dB} + \SI{6}{dB} = \SI{56}{dB}
		\end{equation*}
		\begin{itemize}
			\item Output signal power: $\SI{-73}{dBm} + L_{G,\text{total}} = \SI{-17}{dBm}$
			\item Output noise power: $\SI{-111}{dBm} + L_{G,\text{total}} + L_{F,\text{total}} = \SI{-48.7}{dBm}$
			\item Output SNR: $\SI{38}{dBm} - L_{F,\text{total}} = \SI{31.7}{dB}$
		\end{itemize}
	
		\textbf{Conclusion:}
		\begin{itemize}
			\item Placing the low noise amplifier in front of the chain, gives a much better ($\SI{16.1}{dB}$) SNR. The gain is equal.
			\item The overall noise figure is dominated by the first component in the chain.
			\item In this constellation, first component is a low noise amplifier with high gain and low noise figure.
			\item The high gain scales down the noise contribution of the components following in the chain.
			\item Therefore, it is always feasible to place a low noise amplifier (high gain, low noise) to the front of the chain.
		\end{itemize}
	\end{tasks}
\end{solution}

\begin{question}[subtitle={Python Programming: Cross-correlation}]
Solve all tasks in Python!

Three signal samples are provided along with this exercise. Which one correlates most with the Gauss pulse given in the Python template file?

Make plots and decide from them!
\end{question}

\begin{solution}
\end{solution}



%%%%%%%%%%%%%%%%%%%%%%%%%%%%%%%%%%%%%%%%%%%%%%%%%%%%%%%%%%%%%%%%%%%%%%%%%%%%%%%
%\begin{question}[subtitle={Decibel}]
%	\begin{tasks}
%	\end{tasks}
%\end{question}
%
%\begin{solution}
%	\begin{tasks}
%	\end{tasks}
%\end{solution}
