\documentclass[%
	a4paper,%				A4 Papier
	twoside,%				einseitig (linker und rechter Seitenrand sind gleich groß)
	bibliography=totocnumbered,%	Literaturverzeichnis nummeriert mit ins
	                           % Inhaltsverzeichnis einfügen
	numbers=noenddot,%		hinter der Gliederungsnummer soll kein Punkt gesetzt werden (siehe Duden)
	parskip=half,%			europäischer Satz mit Abstand zwischen Absätzen
	headsepline,%			Linie nach Kopfzeile
	footsepline,%			Linie vor Fusszeile
	headings=small,%		kleine Überschriften
	12pt,%					grössere Schrift, besser lesbar am Bildschrim
]{scrreprt}

%%%%%%%%%%%%%%%%%%%%%%%%%%%%%%%%%%%%%%%%%%%%%%%%%%%%%%%%%%%%%%%%%%%%%%%%%%%
% Language and fonts
\usepackage[UKenglish]{babel}
\usepackage[utf8]{inputenc}
\usepackage[T1]{fontenc}
\usepackage{lmodern}
\usepackage{microtype}
\usepackage{array}

%%%%%%%%%%%%%%%%%%%%%%%%%%%%%%%%%%%%%%%%%%%%%%%%%%%%%%%%%%%%%%%%%%%%%%%%%%%
% Graphics

\usepackage{graphicx}
%\graphicspath{{./}{./bilder/export/}}

\usepackage{float}

% TikZ
\usepackage{tikz}
\usepackage{pgf}
\usepackage{pgfplots}
\usepackage{pgfplotstable}
\pgfplotsset{compat=newest}
\pgfplotsset{
	scriptsize/.style={
		width=4.5cm,
		height=,
		legend style={font=\scriptsize},
		tick label style={font=\scriptsize},
		label style={font=\footnotesize},
		title style={font=\footnotesize},
		every axis title shift=0pt,
		max space between ticks=15,
		every mark/.append style={mark size=7},
		major tick length=0.1cm,
		minor tick length=0.066cm,
	},
}
\pgfplotsset{legend cell align=left}
\pgfplotsset{xmajorgrids}
\pgfplotsset{ymajorgrids}
\pgfplotsset{scale only axis}
\pgfplotsset{every axis plot/.append style={line width=1pt}}
\addto\extrasngerman{\pgfplotsset{/pgf/number format/.cd,set decimal separator={{{,}}}}}
\pgfplotsset{/pgf/number format/.cd,1000 sep={\,}}
\usetikzlibrary{positioning}
\usetikzlibrary{shapes.arrows}
\usetikzlibrary{shapes,arrows}
\usetikzlibrary{decorations.pathreplacing}
\usepackage{pgf-umlsd}

% Circuits
\usepackage[european]{circuitikz}

% Custom TikZ blocks
\tikzset{
	block/.style={
		rectangle,
		align=center,
		minimum height=1cm,
		inner sep=.5cm,
		rounded corners=.15cm
	}
}

% Colours
\usepackage{xcolor}
\usepackage{color}

% Subfigures
\usepackage{subfig}

%%%%%%%%%%%%%%%%%%%%%%%%%%%%%%%%%%%%%%%%%%%%%%%%%%%%%%%%%%%%%%%%%%%%%%%%%%%
% Symbols

% Mathematics
\usepackage{amsmath}
\usepackage{amssymb}
\usepackage{bm}

%%%%%%%%%%%%%%%%%%%%%%%%%%%%%%%%%%%%%%%%%%%%%%%%%%%%%%%%%%%%%%%%%%%%%%%%%%%
% Symbols

\usepackage{rotating}

% Linien für Tabellen
\usepackage{booktabs}

%\usepackage{multirow}
%
%\usepackage{longtable}
%
%\usepackage{bibgerm}
%
%\usepackage{csquotes}
%
%\usepackage[normalem]{ulem}

% Formatierung von Zahlen mit Einheiten
\usepackage[load-configurations=binary]{siunitx}
\sisetup{per-mode=fraction,mode=math}
\addto\extrasngerman{\sisetup{output-decimal-marker={,}}}
\addto\extrasenglish{\sisetup{output-decimal-marker={.}}}
\addto\extrasngerman{\sisetup{range-phrase={ bis~}}} 
\addto\extrasenglish{\sisetup{range-phrase={ to~}}}

%\usepackage{xspace}
%
%\usepackage{xfrac}
%
%\usepackage{bigfoot}
%

%%%%%%%%%%%%%%%%%%%%%%%%%%%%%%%%%%%%%%%%%%%%%%%%%%%%%%%%%%%%%%%%%%%%%%%%%%%
% Page layout
\usepackage{setspace}
\onehalfspacing

\usepackage[a4paper, margin=2.5cm, headheight=22pt]{geometry}

%%%%%%%%%%%%%%%%%%%%%%%%%%%%%%%%%%%%%%%%%%%%%%%%%%%%%%%%%%%%%%%%%%%%%%%%%%%
% Formatting

\usepackage{adjustbox}

%\usepackage[english]{nomencl}
%\makenomenclature
%\usepackage{etoolbox}
%
%% Abkürzungsverzeichnis
%%\usepackage{acronym}
%\usepackage[printonlyused]{acronym}
%
%\renewcommand{\topfraction}{0.8}
%\renewcommand{\bottomfraction}{0.33}
%\renewcommand{\floatpagefraction}{0.66}
%\renewcommand{\textfraction}{0.10}
%
%\usepackage[numbers]{natbib}
%
%\usepackage[
%	pdftitle={Indoor Localization of Ultra Wide Band Beacons using Synchronized Receivers},
%	pdfauthor={Philipp Le},
%	pdfcreator={LaTeX with hyperref and KOMA-Script},
%	pdfsubject={},
%	pdfkeywords={},
%	pdfstartview={Fit},
%	hidelinks]{hyperref}
%
%\usepackage[xindy,numberedsection,section=section,toc]{glossaries}
%\makeglossaries
%\GlsSetXdyCodePage{duden-utf8}
%\usepackage[xindy]{imakeidx}
%\makeindex
%
%\usepackage{pdflscape}
%
%\usepackage{pdfpages}
%
%\usepackage{listings}
%
%\usepackage[subfigure]{tocloft}
%
%\usepackage[bottom]{footmisc}
%\interfootnotelinepenalty=10000
%
%% Fußnotennummerierung nicht in jedem Kapitel neu beginnen
%\let\counterwithout\relax
%\let\counterwithin\relax
%\usepackage{chngcntr}
%\counterwithout{footnote}{chapter}
%%\usepackage{remreset}
%%\makeatletter
%%\@removefromreset{footnote}{chapter}
%%\makeatother
%
%% \emph{} fett darstellen
%%\makeatletter
%%\DeclareRobustCommand{\em}{%
%%  \@nomath\em \if b\expandafter\@car\f@series\@nil
%%  \normalfont \else \bfseries \fi}
%%\makeatother
%	
%%\newcommand{\varDatum}{xx.xx.2018}
%\newcommand{\varAbgabeDatum}{21.08.2019}
%
%
%%\newcommand{\vect}[1]{\boldsymbol{\vec{\mathbf{#1}}}}
%\newcommand{\vect}[1]{\vec{\bm{#1}}}
%\newcommand{\cmplxvect}[1]{\vect{\underline{#1}}}
%\newcommand{\mat}[1]{\bm{\mathrm{#1}}}
%
%\newcommand{\E}{\mathrm{E}}
%\newcommand{\Var}{\mathrm{Var}}
%\newcommand{\Cov}{\mathrm{Cov}}
%\def\j{\mathsf{j}}
%
%\newlistof{todos}{mcf}{To Do}
%\newcommand{\todo}[1]{\texttt{\textbf{\#\# TODO \#\# #1 \#\#}} \addcontentsline{mcf}{todos}{#1}}

