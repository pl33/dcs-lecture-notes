% SPDX-License-Identifier: CC-BY-SA-4.0
%
% Copyright (c) 2020 Philipp Le
%
% Except where otherwise noted, this work is licensed under a
% Creative Commons Attribution-ShareAlike 4.0 License.
%
% Please find the full copy of the licence at:
% https://creativecommons.org/licenses/by-sa/4.0/legalcode

\phantomsection
\addcontentsline{toc}{chapter}{Abstract}
\chapter*{Abstract}

The course covers the properties and functions of digital communication systems. The first part provides a recap on signal description in the time-domain and frequency-domain, stochastic and deterministic signal, and digital signal processing. Based on these foundations, various aspects of modern digital communication are considered. The concept of noise is introduced. System components of communication systems (mixers, filters, etc.) are discussed. Spread spectrum and channel access technologies are introduced (OFDM, CDMA, …). Basics of information theory and coding are considered as well as challenges of communication over wireless transmission channels.

\textit{Keywords} -- digital communication, communication system, signal processing, spread spectrum, multiple access, wireless communication, OFDM, CMDA, information theory, coding
